\section{Pr�sentation du projet}
\label{sec:presentation}

\subsection{Contexte}
Le module de G�nie Logiciel et Programmation de L2 nous demandant la r�alisation d'un projet en Java et souhaitant repr�senter un syst�me et son �volution suivant diff�rents stimulis, al�atoires ou non, il paraissait judicieux de s'orienter vers un sujet de ce type.
\newline
Ayant initialement choisi Psychologie et le projet �tant d�j� attribu� � un autre groupe, le choix d'Histoire parut logique.
\subsection{Objectif du projet avec sp�cifications}
L'objectif de ce projet �tait de r�aliser un simulateur d'�v�nements historiques simulant plusieurs peuples et leurs interactions, positives ou n�gatives.
\newline
Nous souhaitions mettre en place un syst�me d'�v�nements, al�atoires ou non, pouvant alt�rer l'�tat et les statistiques des peuples, et ainsi rendre chaque simulation unique.
\newline
Il semblait int�ressant d'ajouter �galement un syst�me de graphiques permettant la mise en avant d'un aspect p�dagogique, et facilitant la cr�ation d'uchronies.
\subsection{Organisation}
\begin{itemize}
\bulletitem Matteo Staiano : Interface Graphique
\bulletitem Mathieu Hannoun : Conception backend
\bulletitem Commun : R�flexion algorithmique, debugging, compte-rendus et �l�ments de livraison finale.
\end{itemize}
\subsection{Environnements de travail et outils utilis�s}
\begin{itemize}
\bulletitem Programmation Java : Eclipse, Intellij IDEA
\bulletitem VCS : GitHub
\bulletitem Production du rapport \LaTeX{} : TeXnicCenter
\end{itemize}