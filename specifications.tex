\section{Sp�cifications et fonctionnalit�s finales}
\label{sec:specifications}

\subsection{Backend}
\paragraph{Peuple}
Chaque peuple commence avec des attributs principaux bien sp�cifiques dont d�coulent des attributs secondaires.
\subparagraph{Attributs principaux\protect\footnote{D�tails des relations entre attributs dans le sch�ma en derni�re page}}
\begin{itemize}
\bulletitem Ressources
\bulletitem Population
\bulletitem Agressivit�
\bulletitem Education
\bulletitem Territoire
\end{itemize}
\subparagraph{Attributs secondaires}
\begin{itemize}
\bulletitem A
\bulletitem B
\bulletitem C
\end{itemize}
\smallbreak
Les diff�rents peuples peuvent avoir des relations avec les autres sous deux formes : la guerre et le commerce. Le d�clenchement de ces relations ne d�pend que des diff�rents attributs secondaires.
\paragraph{Guerre et commerce}
Chaque tour, des guerres et des liens commerciaux commencent, ou non, entre les peuples, en fonction de leurs caract�ristiques secondaires. Ces relations n'influent que sur les attributs principaux.
\newline
La guerre est co�teuse en population pour les deux peuples, mais apporte richesse et territoire au peuple disposant de la plus grande puissance militaire.
\newline
Le commerce apporte un b�n�fice mutuel aux deux peuples, mais plus un peuple dispose de puissance politique et plus il sera capable de tirer b�n�fice d'un commerce avec un autre.